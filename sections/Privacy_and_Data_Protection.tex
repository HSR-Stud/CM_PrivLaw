% !TeX spellcheck = en_US
\section{Privacy and data protection}
\subsection{Why privacy and data protection?}
\begin{compactitem}
	\item Humans are social (public) - but individuals too (privacy)! Steady conflict between these	2 natures!
	\item If you follow the idea that humans are independent - who decides what and how	information about you can be used by others?
	\item Privacy is a human right. All modern democracy are based upon.
	\item „Data Protection“ doesn’t mean protection data - it means protecting people!
	\item For most companies: huge reputation risk if they misuse personal data!
	\item Disregarding Data Protection as business model? Social Media…
	\item Personal data in wrong hands may harm the person in various forms.
	\item Privacy and the „right to forget“ is essential to develop personally!
	\item An owner with big amount of personal data gets into a powerful position without control (Facebook, Google).
	\item One of the main tasks of the state is to protect his citizens. How can he protect them from misusing personal data?
	\item Personal Data has a financial worth. Who is entitled to use (t)his data? The data is „sold“ once but economically used several times!
\end{compactitem}

\subsection{Data protection}
\begin{compactitem}
	\item Data Protection = Protection of Personal Integrity against infringements!
	\item Established in the second half of 20th century. No clear definition.
	\item Upon definition - Data Protection =
	\begin{compactitem}
		\item protection of the informational self-determination
		\item protection of the personality in the data processing
		\item protection of the privacy
		\item protection from abusive data processing
	\end{compactitem}
\end{compactitem}

\subsection{Proportionality of data processing}
Main Question: Do the reason of the collected personal data justify the penetration in ones privacy?
\begin{compactitem}	
	\item Which personal data exactly?
	\item Who has access to this data?
	\item How long will the personal data be stored?
\end{compactitem}

\subsection{Legal base}
\begin{compactitem}
	\item Swiss Constitution Art. 13 (Bundesverfassung)
	\item Swiss Federal Act on Data Protection (Bundesgesetz über den Datenschutz, DSG). Currently on revision! (adapting to EU-law)
	\item Ordinance to FADP (Verordnung zum Bundesgesetz über den Datenschutz, VDSG)
	\item Several data-protection rules in a wide area of acts (i.e. CO)
	\item Canton and communal: several Data Protection Acts and its by-laws
	\item International: EU-GDPR (General Data Protection Regulation - DSGVO)!
\end{compactitem}

\subsubsection{Protection of integrity/legal personality - ART. 28 CC (ZGB)}
\begin{compactenum}
	\item Any person whose personality rights are unlawfully infringed may petition the court for protecting against all those causing the infringement.
	\item An infringement is unlawful unless it is justified by the consent of the person whose rights are infringed or by an overriding private or public interest or by law .
\end{compactenum}

\subsubsection{Area of Application - ART. 2 DSG}
\begin{compactenum}
	\item This Act applies to the processing of data pertaining to natural persons and legal persons by:
	\begin{compactenum}
		\item \textbf{private persons};
		\item federal bodies.
	\end{compactenum}
	\item \textbf{It does not apply to}:
	\begin{compactenum}
		\item personal data that is processed \textbf{by a natural person exclusively for personal use and which is not	disclosed to outsiders};
		\item deliberations of the Federal Assembly and in parliamentary committees;
		\item pending civil proceedings, criminal proceedings, international mutual assistance proceedings and proceedings under constitutional or under administrative law, with the exception of administrative proceedings of first instance;
		\item public registers based on private law;
		\item personal data processed by the International Committee of the Red Cross.
	\end{compactenum}
\end{compactenum}

\subsubsection{Terms and Definitions - ART. 3 DSG}
The following definitions apply:
\begin{compactenum}	
	\item \textbf{personal data (data)}: all information relating to an identified or identifiable person;
	\item \textbf{data subjects:} natural or legal persons whose data is processed;
	\item \textbf{sensitive personal data:} data on:
	\begin{compactenum}	
		\item religious, ideological, political or trade union-related views or activities,
		\item health, the intimate sphere or the racial origin,
		\item social security measures,
		\item administrative or criminal proceedings and sanctions;
	\end{compactenum}
	\item \textbf{personality profile:} a collection of data that permits an assessment of essential characteristics of the personality of a natural person;
	\item \textbf{processing:} any operation with personal data, irrespective of the means applied and the procedure, and in particular the collection, storage, use, revision, disclosure, archiving or destruction of data;
	\item \textbf{disclosure:} making personal data accessible, for example by permitting access, transmission or publication;
	\item \textbf{data file:} any set of personal data that is structured in such a way that the data is accessible by data subject;
	\item \textbf{federal bodies:} federal authorities and services as well as persons who are entrusted with federal public tasks;
	\item \textbf{controller of the data file:} private persons or federal bodies that decide on the purpose and content of a data file;
\end{compactenum}

\subsubsection{Principles - ART. 4 DSG}
\begin{compactenum}	
	\item Personal data may only be processed lawfully.
	\item Its processing must be carried out in good faith and must be proportionate.
	\item Personal data may only be processed for the purpose indicated at the time of collection, that is evident from the circumstances, or that is provided for by law.
	\item The collection of personal data and in particular the purpose of its processing must	be evident to the data subject.
	\item If the consent of the data subject is required for the processing of personal data, such consent is valid only if given voluntarily on the provision of adequate information. Additionally, consent must be given expressly in the case of processing of sensitive personal data or personality profiles.
\end{compactenum}

\subsubsection{Accuracy of personal data - ART. 5 DSG}
\begin{compactenum}	
	\item Anyone who processes personal data \textbf{must make certain that it is correct}. He must take all reasonable measures to ensure that data that is incorrect or incomplete in view of the purpose of its collection is either \textbf{corrected or destroyed}.
	\item \textbf{Any data subject may request that incorrect data be corrected.}
\end{compactenum}

\subsubsection{Information/data security - ART. 7 DSG}
\begin{compactenum}	
	\item Personal data must be protected against \textbf{unauthorized processing} through \textbf{adequate technical and organizational measures}.
	\item The Federal Council issues detailed provisions on the	minimum standards for data security.
\end{compactenum}

\subsubsection{Cross-border disclosure - ART. 6 DSG}
\begin{compactenum}	
	\item Personal data \textbf{may not be disclosed abroad} if the privacy of the data subjects \textbf{would be seriously endangered thereby}, in particular due to the absence of legislation that guarantees adequate protection.
	\item In the absence of legislation that guarantees adequate protection, personal data may be disclosed abroad only if: ...
\end{compactenum}	

\subsubsection{Right to information - ART. 8 DSG}
\begin{compactenum}	
	\item Any person may request information from the controller of a data file as to whether data concerning them is being processed.
	\item The controller of a data file \textbf{must notify the data subject}:
	\begin{compactenum}	
		\item \textbf{of all available data} concerning the subject in the data file, including the available information on the \textbf{source of the data};
		\item \textbf{the purpose of} and if applicable \textbf{the legal basis} for the processing as well as the \textbf{categories of the personal data} processed, the other parties involved with the file and \textbf{the data recipient}.
	\end{compactenum}	
	\item The controller of a data file may arrange for data on the health of the data subject to be communicated by a doctor designated by the subject.
	\item If the controller of a data file has personal data processed by a third party, \textbf{the controller remains under an obligation to provide information}. The third party is under an obligation to provide information if he does not disclose the identity of the controller or if the controller is not domiciled in Switzerland.
	\item The information must normally be provided in \textbf{writing, in the form of a printout or a photocopy, and is free of charge}. The Federal Council regulates exceptions.
	\item \textbf{No one may waive the right to information in advance.}
\end{compactenum}
\textbf{Right to be informed =} 
\begin{compactitem}
	\item Who is authorized? All judicious persons
	\item Who is legally obligated? Owner of data collection
	\item Form of request? None (oral or written)
	\item Range of personal information? All - with exceptions of Art. 2 Abs. 2 DSG and Art. 9 DSG
	\item Deadline to inform the requester? In written form - or upon consent per insight - generally within 30 days!
	\item Fees? Generally free of charge, exceptionally CHF 300.- (must have been announced previously)
\end{compactitem}

\subsubsection{Duty to provide information on the collection of sensitive personal data and personality profiles - ART. 14 DSG}
\begin{compactenum}	
	\item The controller of the data file \textbf{is obliged to inform the data subject} of the collection \textbf{of sensitive personal data or personality profiles}; this duty to provide information also applies where the data is collected from third parties.
	\item The data subject must be notified as a minimum of the following:
	\begin{compactenum}	
		\item the \textbf{controller of the data file};
		\item the \textbf{purpose} of the processing;
		\item the \textbf{categories of data recipients} if a disclosure of data is planned.
	\end{compactenum}
\end{compactenum}

\subsubsection{Breach of obligations to provide information, to register ot to cooperate - ART. 34 DSG}
\begin{compactenum}	
	\item On complaint,\textbf{ private persons are liable to a fine} if they:
	\begin{compactenum}	
		\item breach their obligations under Articles 8–10 and 14, in that they \textbf{willfully provide false or incomplete information}; or
		\item wilfully fail:
		\begin{compactenum}	
			\item \textbf{to inform the data subject in accordance with Article 14} paragraph 1, or
			\item \textbf{to provide information required under Article 14} paragraph 2.
		\end{compactenum}
	\end{compactenum}
	\item Private persons \textbf{are liable to a fine if they willfully}:
	\begin{compactenum}	
		\item \textbf{fail to provide information} in accordance with Article 6 paragraph 3 or to declare files in accordance with Article 11a or who in doing so willfully provide false information; or
		\item \textbf{provide the Commissioner with false information} in the course of a case investigation (Art. 29) or who refuse to cooperate.
	\end{compactenum}
\end{compactenum}

\subsubsection{Breach of professional confidentiality - ART. 35 DSG}
\begin{compactenum}	
	\item Anyone who without authorization \textbf{willfully discloses confidential, sensitive personal data or personality profiles} that have come to their knowledge in course of their professional activities where such activities require the knowledge of such data, \textbf{on complaint, liable to a fine}.
\end{compactenum}

\subsubsection{Data processing by third parties - ART. 10A DSG}
\begin{compactenum}	
	\item The processing of personal data may be assigned to third parties by agreement of by law if:
	\begin{compactenum}	
		\item the data is processed \textbf{only in the manner permitted for the instructing party itself}; and
		\item \textbf{it is not prohibited by a statutory or contractual duty of confidentiality}.
	\end{compactenum}
	\item The instructing party must in particular \textbf{ensure that the third party guarantees data security}.
	\item Third parties may claim the same justification as the instructing party.
\end{compactenum}

\subsubsection{Employment and data protection - ART. 328B OR}
„The employer may data about the employee only process as it concerns his qualification for the employment or are inevitable for the execution of the employment. The regulations of the Swiss Data Protection Act are applicable.“

\subsubsection{EU-law}
\begin{compactitem}	
	\item EU-GDPR (General Data Protection Regulation - DSGVO)
	\item Directly applicable for all swiss companies if they address to customers in the EU (by using \euro\ or language) and are collecting personal data! (even when using Google-Analytics!)
	\item Stronger requirements i.e. in informing the data subjects about the reason, the categories, the recipients and the rights they have when a company collects personal data.
	\item A responsive have to be named in the public
	\item Draconic fines (up to 20 Mio. \euro)!
\end{compactitem}