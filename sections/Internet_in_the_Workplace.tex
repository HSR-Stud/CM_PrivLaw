% !TeX spellcheck = en_US
\section{Internet in the workplace}
\subsection{Main interests}
\textbf{Employers:}
\begin{compactitem}
	\item Employees use their working hours mostly in favor of the	employer.
	\item The employer has constant access to the (by copyrights protected) working results of the employees.
	\item Employee keeps secrets and sensitive information confidential.
	\item Employee doesn’t mix professional and private data.
	\item Employee acts loyal and don’t misuse the infrastructure.
\end{compactitem}
\textbf{Employees:}
\begin{compactitem}
	\item Access to the information necessary to fulfill the job. „Blockers“ not too strict!
	\item Not to be disconnected from the private world.
	\item Employer respects the privacy at work.
	\item Employer protects personal data.
\end{compactitem}

\subsection{Controlling the communication}
\begin{compactitem}
	\item Permanent and systematic control of the communication (internet, email, phone, chats etc.) is illegal!
	\item If previously announced and the control is limited to a short time: control samples and individual control is legal.
	\item Criminal investigation is legal.
	\item It’s wise to let sign a „Code of Conduct“.
	\item Potential conflicts between interests of security (i.e. video), production control (Product Liability) and the prohibition of observing the employees!
\end{compactitem}

\subsubsection{Typical problems}
\begin{compactitem}
	\item Employees use their own devices („BYOD“)
	\item Home office - obligations for the employer?
	\item 24/7 h availability of the employee?
	\item Recording phone calls for „educational and quality reasons“?
	\item Mixing professional and private data: questions about the release of saved private data.
\end{compactitem}

\subsection{Legal base}
\subsubsection{Employment and data protection - ART. 328B OR}
„The employer may data about the employee \textbf{only} process as it concerns\textbf{ his qualification for the employment} or are \textbf{inevitable for the execution of the employment}. The regulations of the Swiss Data Protection Act are applicable.“

\textit{Der Arbeitgeber darf Daten über den Arbeitnehmer nur bearbeiten, soweit sie dessen Eignung für das Arbeitsverhältnis betreffen oder zur Durchführung des Arbeitsvertrages erforderlich sind. Im Übrigen gelten die Bestimmungen des Bundesgesetzes vom 19. Juni 1992 über den Datenschutz.}

\subsubsection{Observing the employers - ART. 26 ArGV 3}
\textit{\begin{compactenum}
	\item Überwachungs- und Kontrollsysteme, die das Verhalten der Arbeitnehmer	am Arbeitsplatz überwachen sollen, dürfen nicht eingesetzt werden.
	\item Sind Überwachungs- oder Kontrollsysteme aus anderen Gründen erforderlich, sind sie insbesondere so zu gestalten und anzuordnen, dass die Gesundheit und die Bewegungsfreiheit der Arbeitnehmer dadurch nicht beeinträchtigt werden.
\end{compactenum}}

\section{Whistleblowing}
A whistleblower is a person who exposes any kind of information or activity that is deemed illegal, unethical or not correct within an organization that is either private or public.
\begin{compactitem}
	\item It might be illegal to disclose such information, but ethically required!
	\item Big risks for both - the whistleblower as for the related organization!
	\item Illegal = breaking the law
	\item Illegitimate = breaking of international regulations and directives
	\item Unethical = unmoral and unethical practices
\end{compactitem}

\subsection{Legal base}
\begin{compactitem}
	\item If an employee finds out illegal activities in his field of responsibility he has to inform his superior (duty of good faith, Art. 321a Abs. 1 OR).
	\item Employees of a public authority are obliged to inform the superiors about illegal	activities (Art. 22a BPG).
	\item If the whistleblower gets no attention he’s only allowed to inform the public when several conditions are fulfilled!
	\item Corporate Governance and Compliance/Risk Management requires a save reporting	organization (Art. 716a Abs. 1 OR). Eventually Sarbanes-Oxley Act (SOX)
	\item It has to ensured that the whistleblower keeps his anonymity but (back-)communication	is possible with him.
\end{compactitem}

\subsubsection{Legal protection}
\begin{compactitem}
	\item Freedom of expression is protected by Art. 10 EMRK and Art. 16 BV but in practice real risk of criminal prosecution.
	\item There had been a draft for a new law (2013) but huge critics in the consultation and by the parliament. Still under development. From	humanrights.ch and other organisations are Ombudsmen demanded.
	\item In cases of abusive termination of employment agreements exists a	rudimentary protection - a penalty of max. 6 monthly salaries (Art. 336 OR).
\end{compactitem}
